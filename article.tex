%%%%%%%%%%%%%%%%%%%%%%%%%%%%%%%%%%%%%%%%%
% Journal Article
% LaTeX Template
% Version EG-1.5 (05/07/2018)
%
% This template is a modified version of a template downloaded from
% http://www.LaTeXTemplates.com, originally created by Frits Wenneker
% (http://www.howtotex.com) with extensive modifications by Vel
% (vel@LaTeXTemplates.com).
%
% This version further modified by Eduard Grebe (ed@eduardgrebe.net). A few
% elements of the PLOS submission style have been incorporated
%
% Available from https://github.com/eduardgrebe/preprint-latex-template
%
% License:
% CC BY-NC-SA 3.0 (http://creativecommons.org/licenses/by-nc-sa/3.0/)
%
% This template is designed to be compiled with XeLaTeX
%
%%%%%%%%%%%%%%%%%%%%%%%%%%%%%%%%%%%%%%%%%

%----------------------------------------------------------------------------------------
%	PACKAGES AND OTHER DOCUMENT CONFIGURATIONS
%----------------------------------------------------------------------------------------

\documentclass[a4paper,10pt]{article} %twocolumn 12pt

\usepackage{blindtext} % Package to generate dummy text throughout this template

% Uncomment the following 2 lines for compilation
%\usepackage[sc]{mathpazo} % Use the Palatino font
%\usepackage[T1]{fontenc} % Use 8-bit encoding that has 256 glyphs
\usepackage[utf8]{inputenc}

% The next 3 lines require compilation with XeLaTeX and installation of
% Baskerville and Menlo (default on macOS). To compile with pdflatex comment out
% the lines
\usepackage{fontspec} % \usepackage[no-math]{fontspec}
\setmainfont[Mapping=tex-text]{Baskerville}
\setmonofont[Scale=0.8]{Menlo}
%\usepackage{unicode-math}
%\setmathfont[math-style=ISO,bold-style=ISO,vargreek-shape=TeX]{TeX Gyre Pagella Math}

% This should use a maths font that matches Palatino
%\usepackage{mathpazo}


%\linespread{1.05} % Line spacing - Palatino needs more space between lines
%\usepackage{microtype} % Slightly tweak font spacing for aesthetics

\usepackage[english]{babel} % Language hyphenation and typographical rules

\usepackage[hmarginratio=1:1,top=32mm,columnsep=20pt]{geometry} % Document margins
\usepackage[hang, small,labelfont=bf,up,textfont=it,up]{caption} % Custom captions under/above floats in tables or figures
\usepackage{booktabs} % Horizontal rules in tables

\usepackage{lettrine} % The lettrine is the first enlarged letter at the beginning of the text

\usepackage{enumitem} % Customized lists
\setlist[itemize]{noitemsep} % Make itemize lists more compact

\usepackage{abstract} % Allows abstract customization
\renewcommand{\abstractnamefont}{\normalfont\bfseries} % Set the "Abstract" text to bold
\renewcommand{\abstracttextfont}{\normalfont\small\itshape} % Set the abstract itself to small italic text

\usepackage{titlesec} % Allows customization of titles
%\renewcommand\thesection{\Roman{section}} % Roman numerals for the sections
%\renewcommand\thesubsection{\roman{subsection}} % roman numerals for subsections
%\titleformat{\section}[block]{\large\scshape\centering}{\thesection.}{1em}{} % Change the look of the section titles
%\titleformat{\subsection}[block]{\large}{\thesubsection.}{1em}{} % Change the look of the section titles

\usepackage{fancyhdr} % Headers and footers
\pagestyle{fancy} % All pages have headers and footers
\fancyhead{} % Blank out the default header
\fancyfoot{} % Blank out the default footer
\fancyhead[C]{Running title $\bullet$ \today} % Custom header text % $\bullet$ Vol. XXI, No. 1
\fancyfoot[RO,LE]{\thepage} % Custom footer text

\usepackage{titling} % Customizing the title section

\usepackage{authblk}

\usepackage{hyperref} % For hyperlinks in the PDF

% Use adjustwidth environment to exceed column width (see example table in text)
\usepackage{changepage}

% Bold and left justify captions, as well as ensure aligned with \adjustwidth
% environment
%\usepackage[aboveskip=1pt,labelfont=bf,justification=centering,singlelinecheck=off]{caption} %raggedright

%\usepackage{array}
% \thickhline command for thick horizontal lines that span the table
%\newcommand\thickhline{\noalign{\global\savedwidth\arrayrulewidth\global\arrayrulewidth 2pt}%
%\hline
%\noalign{\global\arrayrulewidth\savedwidth}}



%----------------------------------------------------------------------------------------
%	TITLE SECTION
%----------------------------------------------------------------------------------------

\setlength{\droptitle}{-4\baselineskip} % Move the title up

\pretitle{\begin{center}\Huge\bfseries} % Article title formatting
\posttitle{\end{center}} % Article title closing formatting

\title{Article Title} % Article title

\author[1†]{First Author}
\author[2†*]{Second Author}
\author[3]{Third Author}
\affil[1]{First Affiliation}
\affil[2]{Second Affiliation}
\affil[3]{Third Affiliation}
\affil[†]{These authors contributed equally.}
\affil[*]{\href{mailto:eduardgrebe@sun.ac.za}{eduardgrebe@sun.ac.za}}
% \author{%
% \textsc{John Smith}\thanks{A thank you or further information} \\[1ex] % Your name
% \normalsize University of California \\ % Your institution
% \normalsize \href{mailto:john@smith.com}{john@smith.com} % Your email address
% %\and % Uncomment if 2 authors are required, duplicate these 4 lines if more
% %\textsc{Jane Smith}\thanks{Corresponding author} \\[1ex] % Second author's name
% %\normalsize University of Utah \\ % Second author's institution
% %\normalsize \href{mailto:jane@smith.com}{jane@smith.com} % Second author's email address
% }
\date{\today} % Leave empty to omit a date
\renewcommand{\maketitlehookd}{%
\begin{abstract}
\noindent \blindtext % Dummy abstract text - replace \blindtext with your abstract text
\end{abstract}
}

%----------------------------------------------------------------------------------------

\begin{document}

% Print the title
\maketitle

%----------------------------------------------------------------------------------------
%	ARTICLE CONTENTS
%----------------------------------------------------------------------------------------

\section{Introduction}

\lettrine[nindent=0em,lines=3]{L} orem ipsum dolor sit amet, consectetur adipiscing elit.
\blindtext % Dummy text

\begin{eqnarray}
\label{eq:kassanjee}
\lambda = \frac{P_H \cdot (P_{R|+}-\epsilon_T)}{(1 - P_H) \cdot (\Omega_T - \epsilon_T \cdot T)}
\end{eqnarray}

\blindtext % Dummy text

%------------------------------------------------

\section{Methods}

Maecenas sed ultricies felis. Sed imperdiet dictum arcu a egestas.
\begin{itemize}
\item Donec dolor arcu, rutrum id molestie in, viverra sed diam
\item Curabitur feugiat
\item turpis sed auctor facilisis
\item arcu eros accumsan lorem, at posuere mi diam sit amet tortor
\item Fusce fermentum, mi sit amet euismod rutrum
\item sem lorem molestie diam, iaculis aliquet sapien tortor non nisi
\item Pellentesque bibendum pretium aliquet
\end{itemize}
\blindtext % Dummy text

Text requiring further explanation\footnote{Example footnote}.

%------------------------------------------------

\section{Results}

\begin{table}
\caption{{\bf Example table}}
\centering
\begin{tabular}{llr}
\toprule
\multicolumn{2}{c}{Name} \\
\cmidrule(r){1-2}
First name & Last Name & Grade \\
\midrule
John & Doe & $7.5$ \\
Richard & Miles & $2$ \\
\bottomrule
\end{tabular}
\end{table}

\begin{table}[!ht]
\begin{adjustwidth}{-20mm}{0mm} % Comment out/remove adjustwidth environment if table fits in text column.
\begin{flushleft}\caption{
{\bf ‘Average incidence’ estimates by age group using the biomarker and synthetic cohort methods.}}\end{flushleft}
\centering
\begin{tabular}{|l|c|c|c|c|c|c|}
\hline
% \multicolumn{1}{|c|}{\bf Age group} & {\bf Males}         & {\bf Females}       & {\bf Total} \\
%                 & P.E. (95\% CI)      & P.E. (95\% CI)      & P.E. (95\% CI) \\
% \multicolumn{1}{|c|}{\emph{years}}   & \emph{cases/100PY} & \emph{cases/100PY} & \emph{cases/100PY} \\ \thickhline
                                      & \multicolumn{3}{|c|}{\bf Age-continuous biomarker mothod} & \multicolumn{3}{|c|}{\bf Synthetic cohort method}    \\ \hline
\multicolumn{1}{|c|}{\bf Age group}   & {\bf Males}         & {\bf Females}       & {\bf Total}         & {\bf Males}         & {\bf Females}       & {\bf Total} \\
                                      & P.E. (95\% CI)      & P.E. (95\% CI)      & P.E. (95\% CI)      & P.E. (95\% CI)      & P.E. (95\% CI)      & P.E. (95\% CI) \\
\multicolumn{1}{|c|}{\emph{years}}    & \emph{cases/100PY}  & \emph{cases/100PY}  & \emph{cases/100PY}  & \emph{cases/100PY}  & \emph{cases/100PY}  & \emph{cases/100PY} \\ \hline %\thickhline
[15,20)                               & 0.45 (0.08,0.88)    & 1.92 (1.02,3.05)    & 1.20 (0.73,1.78)    & 0.27 (0.00,0.60)    & 2.90 (2.30,3.48)    & 1.61 (1.24,1.92) \\ \hline
[20,25)                               & 0.71 (0.16,1.38)    & 3.73 (2.41,5.30)    & 2.20 (1.51,2.92)    & 1.59 (1.14,2.05)    & 5.26 (4.53,6.20)    & 3.40 (2.96,3.91) \\ \hline
[25,30)                               & 1.50 (0.40,2.73)    & 3.62 (2.04,5.38)    & 2.52 (1.52,3.45)    & 5.09 (3.36,6.64)    & 5.70 (4.33,7.30)    & 5.38 (4.23,6.65) \\ \hline
[30,35)                               & 2.02 (0.24,4.50)    & 2.66 (0.71,5.47)    & 2.33 (0.99,4.30)    & 4.07 (0.00,11.18)   & 8.53 (2.29,16.43)   & 6.20 (2.47,11.29) \\ \hline
[15,30)                               & 0.81 (0.22,1.45)    & 2.95 (1.98,4.04)    & 1.87 (1.31,2.43)    & 2.00 (1.53,2.46)    & 4.39 (4.00,4.85)    & 3.19 (2.83,3.56) \\ \hline
\end{tabular}
\begin{flushleft} Weighted by susceptible population density.
\end{flushleft}
\label{table2}
\end{adjustwidth}
\end{table}

\blindtext % Dummy text

\begin{equation}
\label{eq:emc}
e = mc^2
\end{equation}

\blindtext % Dummy text

%------------------------------------------------

\section{Discussion}

\subsection{Subsection One}

A statement requiring citation \cite{Schlegel2016}.
\blindtext % Dummy text

\subsection{Subsection Two}

\blindtext % Dummy text

%----------------------------------------------------------------------------------------
%	REFERENCE LIST
%----------------------------------------------------------------------------------------

\bibliography{references}

\bibliographystyle{vancouver}



%----------------------------------------------------------------------------------------

\end{document}
